\section{Sensory Aids}
\subsection{Visual Impairments}

\noindent
\textbf{Paper I} 
\\ \\
\noindent
Bloch, Edward, Yvonne Luo, and Lyndon da Cruz. “Advances in Retinal Prosthesis Systems.” Therapeutic Advances in Ophthalmology 11 (January 17, 2019). https://doi.org/10.1177/2515841418817501.\\ \\

\noindent
\textbf{Keywords} Epiretinal prosthesis, Subretinal prosthesis, Suprachoroidal prosthesis.\\ \\

\noindent
\textbf{Abstract} Retinal prosthesis systems have undergone significant advances in the past quarter century, resulting in the development of several different novel surgical and engineering approaches. Encouraging results have demonstrated partial visual restoration, with improvement in both coarse objective function and performance of everyday tasks. To date, four systems have received marketing approval for use in Europe or the United States, with numerous others undergoing preclinical and clinical evaluation, reflecting the established safety profile of these devices for chronic implantation. This progress represents the first notion that the field of visual restorative medicine could offer blind patients a hope of real and measurable benefit. However, there are numerous complex engineering and biophysical obstacles still to be overcome, to reconcile the gap that remains between artificial and natural vision. Current developments in the form of enhanced image processing algorithms and data transfer approaches, combined with emerging nanofabrication and conductive polymerization techniques, herald an exciting and innovative future for retinal prosthetics. This review provides an update of retinal prosthetic systems currently undergoing development and clinical trials while also addressing future challenges in the field, such as the assessment of functional outcomes in ultra-low vision and strategies for tackling existing hardware and software constraints. \\ \\

\noindent
\textbf{Sensors in use:}
\begin{itemize}
    \item Photodiode array
    \item Camera \\ \\
\end{itemize}

\noindent
\textbf{Summary} This paper reviews identifies and reviews three categories of retinal prosthetic devices. These three categories are epiretinal, subretinal, and suprachoroidal implants that vary based on their placement within the eye. Epiretinal implants have found the most commercial success. The Argus II device (Second Sight Medical Products Inc.) consists of a external glasses-mounted camera and processing unit that records the environment, The recorded image is then sent via RF power modulation to the epiretinal stimulator component that is surgically fixed to the surface of the retina. The advantage of this device is the well-developed surgical procedure, however the drawback lies in the susceptibility of the device to serious adverse events such as the degeneration of the RF link that renders the device instantly nonfunctional. Subretinal protheses are unique in that they directly use light hitting the retina to generate photocurrent using a photodiode array. Though the photocurrent alone is not enough to power retinal stimulation, a simple external power unit makes this device functional without the need for a camera. The Alphs IMS and AMS (Retina Implant AG) are two devices that use the described subretinal implant mechanism. Finally, suprachoroidal prostheses have also been explored by some research groups. The Bionic Vision Australia group designed a 24-channel system that was implanted and tested. The device showed some promise in terms of improvement in visual acuity, however there were complications with the surgical procedure and amount of stimulation current due to array placement relative to retinal neurons. The conclusion was that epiretinal and subretinal implants showed more promise in the long-term as commericial options as compared to suprachoroidal prostheses. \\ \\

%%%%

\textbf{Paper II} 
\\ \\
\noindent
Jeong, Joonsoo, So Hyun Bae, Kyou Sik Min, Jong-Mo Seo, Hum Chung, and Sung June Kim. “A Miniaturized, Eye-Conformable, and Long-Term Reliable Retinal Prosthesis Using Monolithic Fabrication of Liquid Crystal Polymer (LCP).” IEEE Transactions on Biomedical Engineering 62, no. 3 (March 2015): 982–89. https://doi.org/10.1109/TBME.2014.2377197. \\ \\

\noindent
\textbf{Keywords} Biocompatible LCP, Feed-through technology, Flexible polymer-based system, Hermetic metal package. \\ \\

\noindent
\textbf{Background on Packaging} Following the lead of packaging used in established field such as cochlear implants, retinal implants have traditionally used titanium-based encapsulation. The main benefits of this approach is that titanium is both biocompatible and waterproof which makes it an ideal passivation layer for implants. Additionally, for cochlear implants which are generally screw-fixed to a point by a bone, movement of the device is not a major concern which is one of the major drawbacks of using such encapsulation for retinal implants. 
\\ \\

\noindent
\textbf{Abstract} A novel retinal prosthetic device was developed using biocompatible liquid crystal polymer (LCP) to address the problems associated with conventional metal- and polymer-based devices: the hermetic metal package is bulky, heavy, and labor-intensive, whereas a thin, flexible, and MEMS-compatible polymer-based system is not durable enough for chronic implantation. Exploiting the advantageous properties of LCP such as a low moisture absorption rate, thermobonding, and thermoforming, we fabricate a small, light-weight, long-term reliable retinal prosthesis that can be conformally attached on the eye-surface. A LCP fabrication process using monolithic integration and conformal deformation was established enabling miniaturization and a batch manufacturing process as well as eliminating the need for feed-through technology. The functionality of the fabricated device was tested through wireless operation in saline solution. Its efficacy and implantation stability were verified through in vivo animal tests by measuring the cortical potential and monitoring implanted dummy devices for more than a year, respectively. \\ \\

\noindent
\textbf{Sensors in use:}
\begin{itemize}
    \item Visible light camera \\ \\
\end{itemize}

\noindent
\textbf{Summary} This paper explores alternative packaging options for subretinal implants. The traditional mermetically-sealed metal packages are bulky and do not conform well with the spherical wall of the retina. Other polymer based options have failed mainly do the high absorption of water and other substances present within the eye that deteriorate the packaging. The Liquad Crystal Polymer (LCP) packaging proposed in this paper strikes the balance between conformability for user comfort and robustness for long-term usage. The device is currently being testing for long-term reliability, therefore results are still forthcoming. \\ \\

%%%%

\textbf{Paper III} 
\\ \\
\noindent
Kim, Jung-Hun, Ji-Eun Park, and Jong-Min Lee. “3-D Space Visualization System Using Ultrasonic Sensors as an Assistive Device for the Blind.” IEEE Journal of Translational Engineering in Health and Medicine 8 (2020): 1–5. https://doi.org/10.1109/JTEHM.2020.2978842. \\ \\

\noindent
\textbf{Keywords} 3D space visualization system, Assisted living, Ultrasonic transducers. \\ \\

\noindent
\textbf{Abstract} This study proposes a new assistive device for the blind that uses more than one-dimensional data to draw objects. The study aims to convert three-dimensional (3-D) spatial information into sound information using 6-axis and ultrasonic sensors, and to draw a 3-D depiction of the space ahead for the user. Fourteen participants were involved in testing, wherein 4 were visually impaired. Moreover, the male to female ratio was 7:3, with the average age of participants at 28.8 years. An initial sound recognition experiment was designed to assess the device's accuracy through participant use. Recognition rates were 70\% for normal participants and 88\% for the blind participants. Additional experiments expanded the environmental conditions by requiring participants to discern the distances of 10 objects, positioned at both high and low locations. Two different scenarios were employed: stationary and walking scenarios. The stationary distance measurement participants scored an average of 96 points, while the walking participants averaged 81 points. Under the given conditions, this study found that its assistive device for the visually impaired can draw a 3-D space with 88.5\% accuracy. This probability promises a basic level of utility that can assist those with visual impairment in controlled environments, such as hospitals and homes. \\ \\

\noindent
\textbf{Sensors in use:}
\begin{itemize}
    \item Ultrasonic sensor \\ \\
\end{itemize}

\noindent
\textbf{Summary} External visual assistive devices generally consist of a sensor to map the surrounding environment and a feedback device to inform the user of objects to avoid. Much research has been dedicated to the sensor mapping stage with high distance precision and object recognition. However, the communication of environment information to the blind individual is relatively neglected. In this paper, the focus is on the use of three n-octave levels to describe in details the 3D spatial information collected with an ultrasonic sensor. Testing of the sound-based space visualization system results in visually impaired individuals drawing their environment with a 88.5\% of accuracy. \\ \\

\subsection{Auditory Impairments}

\textbf{Paper I} 
\\ \\
\noindent
Eßinger, Till Moritz, Martin Koch, Matthias Bornitz, Nikoloz Lasurashvili, Marcus Neudert, and Thomas Zahnert. “Sensor-Actuator Component for a Floating Mass Transducer-Based Fully Implantable Hearing Aid.” Hearing Research 378 (July 2019): 157–65. https://doi.org/10.1016/j.heares.2019.03.006. \\ \\

\noindent
\textbf{Keywords} Floating Mass Transducer, Piezoelectric sensor, Incudostapedial joint, Cloosed-loop feedback. \\ \\

\noindent
\textbf{Abstract} We propose a novel system based on the Floating Mass Transducer (FMT) to be used as the active component of a fully implantable, Vibrant Soundbridge-like middle ear implant. The new system replaces the external microphone used in the currently available design with an implantable piezoelectric sensor that is inserted into the incudostapedial joint and picks up the vibrations transmitted to the long process of the incus. The FMT is coupled to the round window of the cochlea. We characterize the system by measuring the gain in intracochlear sound pressure using laser Doppler vibrometry at a surgically installed “third window” into the cochlea of six temporal bones. Closed-loop feedback oscillations limit the system's available output. We show that using an adaptive control algorithm, a mean functional gain of up to 40 dB is achieved, which is similar to Soundbridge functional gain. The concept matches the FMT's one-point fixation philosophy and offers several advantages over other designs, namely an easy and time-efficient surgery, reversibility of implantation, and natural hearing for the prospective patient. \\ \\

\noindent
\textbf{Sensors in use:}
\begin{itemize}
    \item Piezoelectric sensor \\ \\
\end{itemize}

\noindent
\textbf{Summary} Cochlear implants are commonly used for treating patients with varying severity of hearing loss. Generally, the device consists of an external microphone that records sound, a processing unit that converts the sound into a control signal, and an implanted portion of the device that stimulate certain parts of the inner ear to simulate that sound. In this paper, the external microphone is omitted in exchange for a Floating Mass Transducer (FMT). The operating principle of the FMT is to record external sound by picking up vibrations in the incudostapedial joint. The vibrations are converted into a sound signal through the use of a peizoelectric sensor that is surgically placed in the joint. The motivation for this is the removal of the bulky external microphone and the push to make the cochlear implant completely implanted in a single surgery. \\ \\

%%%%

\textbf{Paper II} 
\\ \\
\noindent
Li, Yingdan, Fei Chen, Zhuoyi Sun, Junyu Ji, Wen Jia, and Zhihua Wang. “A Smart Binaural Hearing Aid Architecture Leveraging a Smartphone APP With Deep-Learning Speech Enhancement.” IEEE Access 8 (2020): 56798–810. https://doi.org/10.1109/ACCESS.2020.2982212. \\ \\

\noindent
\textbf{Keywords} Binaural hearing aids , Deep Learning speech enhancement, Hearing aid, Smartphone-based hearing aid architecture. \\ \\

\noindent
\textbf{Abstract} This paper presents a smartphone-based binaural hearing aid architecture for improving the speech intelligibility of hearing aid users. The proposed system consists of an earpiece, a smartphone and an application that performs real-time speech enhancement. The speaker's voice, which is picked up by the microphone of the earpiece that is worn on the ear, is transmitted to the smartphone via wireless technology. After the speech intelligibility is improved in real time by the deep learning speech enhancement application, it is returned to the earpiece and generates sound. Deep learning speech enhancement algorithms can be used without performing burdensome calculations on the processors in the hearing aid. The results showed that the average usage of the central processing unit in the smartphone was approximately 26\%, and the signal-to-noise ratios improve by at least 20\%. The presented objective and subjective results show that the proposed method achieves comparatively more noise suppression without distorting the audio. \\ \\

\noindent
\textbf{Sensors in use:}
\begin{itemize}
    \item External hearing aid microphone \\ \\
\end{itemize}

\noindent
\textbf{Summary} The ubiquitous usage of smartphones and the impressive growth in smartphone processing power has allowed for traditional sensory aids to use more complex algorithms. This same approach is used in this paper to implement a real-time deep learning (DL) algorithm to enhance recorded speech that is then played by the hearing aid for the patient. The microphone first records the sound which is then transmitted via bluetooth to the user's smartphone. The smartphone stores the signal in a buffer that is is process by the DL algorithm in real-time. With an average delay of 4-6 ms, the data is then transmitted back to the speaker that plays the sound for the user. Testing of the device showed that the DL algorithim improved signal-to-noise ratio by a minimum amount of 20\%. \\ \\

%%%%

\textbf{Paper III} 
\\ \\
\noindent
Lin, Bor-Shing, Po-Yu Yang, Ching-Feng Liu, Yi-Chia Huang, Chengyu Liu, and Bor-Shyh Lin. “Design of Novel Field Programmable Gate Array-Based Hearing Aid.” IEEE Access 7 (2019): 63809–17. https://doi.org/10.1109/ACCESS.2019.2916723. \\ \\

\noindent
\textbf{Keywords} FPGA-based hearing aid, Direction-of-arrival estimation, Image processing technology, Adaptive beamforming. \\ \\

\noindent
\textbf{Abstract} Hearing loss is one of the most common chronic diseases. For people with hearing loss, communicating with other people, particularly in an environment with considerable background noise, is difficult. Recently, several hearing aids have been developed to improve speech comprehension in a noisy environment. The use of an adaptive beamformer is one of the alternative methods for improving speech intelligibility. However, the adaptive beamformer requires the location of the desired speaker to estimate the time differences of arrival (TDOAs) of speech sources to numerous spatially separated sensors in acoustics. In general, the technique of steered response power source localization was used to estimate the TDOA; however, this technique was easily affected by environmental reverberation. To overcome the aforementioned concern, a novel hearing aid is proposed in this paper. By using an image processing technology, the location of the desired speaker could be manually selected to provide precise information on the TDOA. Moreover, adaptive signal enhancements were implemented in a field-programmable gate array to enhance the speech of interest in real time. The experimental results indicate that the proposed system could improve speech intelligibility in various noisy environments. Therefore, the proposed system may be employed to improve the daily lives of people with hearing the loss in the future. \\ \\

\noindent
\textbf{Sensors in use:}
\begin{itemize}
    \item Wide angle camera
    \item Four-microphone array \\ \\
\end{itemize}

\noindent
\textbf{Summary} Adaptive beamforming is a signal processing technique that improves the signal strength in one direction by combining sounds from an array of different sensors. As a result, the beamforming technique can separate noise and signal using spatial information which is beneficial for improving hearing aid speech intelligibility. The spatial information is traditionally acquired using a Steered Response Power Source Localization (SRPSL) algorithm. This algorithm calculates how far the speaker is based on the power of the signal that is received along with other variables. The disadvantage of SRPSL is that it the algorithm is susceptible to environmental factors including reverberations picked up by the microphones as well as poor signal quality. This paper proposes the use of image processing and machine learning algorithms to detect where the speaker is and use this information to calculate time difference of arrival from speech sources. A field-programmable gate array (FPGA) is used to code the logic that will convert live video from a wide-angle camera into spatial information that can be used in the adaptive beamforming technique. After testing, the proposed system improved the signal-to-noise ratio of sound from a noisy environment by ~5dB. \\ \\

%%%%

\textbf{Paper IV} 
\\ \\
\noindent
M. K. Cosetti and S. B. Waltzman, “Cochlear implants: current status and future potential,” null, vol. 8, no. 3, pp. 389–401, May 2011, doi: 10.1586/erd.11.12. \\ \\

\noindent
\textbf{Keywords} Advanced Bionics, HiRes CI, Cochlear Corporation, Nucleus CI, MED-EL, PULSE CI, Speech processing, Electro-acoustic stimulation, Partial Deafness. \\ \\

\noindent
\textbf{Abstract} This article reviews the current status of cochlear implantation in both adults and children, including expanding candidacy groups, bilateral implantation, advances in speech processing software, internal and external device hardware, surgical techniques and outcomes. Promising advances, novel therapies and evolving concepts are also highlighted in terms of their future impact on clinical outcomes. \\ \\

\noindent
\textbf{Summary} Cochlear implantation (CI) is a well-established and accepted treatment for hearing loss. Over 220,000 individuals as of 2011 had CIs. In this paper, the three most common commercial CIs including the HiRes devices (Advanced Bionics), Nucleus devices (Cochlear Ltd.), and MAESTRO devices (MED-EL) are compared on the basis of a number of technical features. Mainly, the focus is on the most recent efforts of these companies to improve their respective devices to get better spatial specificity within the cochlea. A number of factors limit this specificity including inner ear fluid that moves the stimulator, interference between adjacent electrode sites, and the distance between the stimulating electrode and the ganglia neurons. The HiRes CI series uses a Fidelity 120 processing strategy that uses current steering to not only reduce adjacent electrode interference, but to also create "virtual" channels. Additionally, a preprocessing strategy that is implemented prior to the transmission to the implanted stimulator works to reduce noise immediately following recording by the microphone. As a result, the HiRes devices have shown better pitch and tone discrimination which is directly related to improved speech perception. The Nucleus devices also use a processing and preprocessing step. The processing step eliminates duplicated or masked sounds prior to stimulation. The preprocessing step improves the audio source by using an adaptive directional microphone. Going further, a special modiolar-hugging electrode is used that helps physically space out the electrodes in the most optimized manner while maintaining closeness to the ganglia neurons. Finally, the MED-EL MAESTRO line of devices uses simultaneous stimulation of cochlea sections to improve speech perception versus the traditional sequential approach. Similar to the previous device, a custom FlexEAS electrode is used that spaces the electrodes appropriately while having the added benefit of maintaining existing inner ear functionality. Like the Nucleus devices, the MED-EL devices also use titanium stimulators rather than ceramic ones to reduce size and increase usage in pediatric patients. \\ \\

%%%%%%%%%%%%%%%%%%%%%%%%%%

% \tab Of the five basic senses, the senses that are arguably the most significant contributors to a high quality of life are hearing and sight. Due to this, a lot of research and funding has gone into developing sensory aids for individuals who suffer from impairment of either or both of these senses.