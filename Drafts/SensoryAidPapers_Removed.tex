\textbf{Paper I} 
\\ \\
\noindent
Song, Chao, Gil Ben-Shlomo, and Long Que. “A Multifunctional Smart Soft Contact Lens Device Enabled by Nanopore Thin Film for Glaucoma Diagnostics and In Situ Drug Delivery.” Journal of Microelectromechanical Systems 28, no. 5 (October 2019): 810–16. https://doi.org/10.1109/JMEMS.2019.2927232.\\ \\

\noindent
\textbf{Keywords} Biomarkers, Glaucoma diagnostics, In situ drug delivery, Intraocular pressure, Nanopore thin film.\\ \\

\noindent
\textbf{Abstract} In this paper, we report a new power-free multifunctional soft contact lens device that can measure intraocular pressure (IOP), achieve extended drug delivery in situ, and detect glaucoma biomarkers, all within the same device. Experiments demonstrate that the contact lens sensor can detect Interleukin 12p70, one possible biomarker for glaucoma, in a concentration as low as 2 pg/ml in artificial tears. The sustained drug release of the contact lens device can last up to 30 days. In ex vivo tests using cadaver pig eyes, the sensor detected IOP in a range of 10-50 mmHg with excellent repeatability.\\ \\

\noindent
\textbf{Sensors in use:}
\begin{itemize}
    \item Anodic Aluminum Oxide (AAO) Thin Film IOP sensor \\ \\
\end{itemize}

\noindent
\textbf{Summary} This paper describes an AAO thin film nanopore structure that measures IOP and contains a drug delivery system within the contact lens. Light is flashed on to the lens and the reflected light is collected and fed into a spectrometer. The amount of phase shift in the reflected light is indicative of the concentration of the IOP biomarker Ags. The biomarker bonds with the antibody Abs resulting in the phase shift. The motivation for this device is the rapid growth in number of individuals afflicted by glaucoma which is the leading cause of blindness worldwide. The high cost and specialized training required to use a tonometry device for IOP measurement necessitated the design of a low-cost and convenient IOP sensor. \\ \\

%%%%

\textbf{Paper II} 
\\ \\
\noindent
Gajecki, Tom, and Waldo Nogueira. “The Effect of Synchronized Linked Band Selection on Speech Intelligibility of Bilateral Cochlear Implant Users.” Hearing Research 396 (October 2020): 108051. https://doi.org/10.1016/j.heares.2020.108051. \\ \\

\noindent
\textbf{Keywords} Linked band selection, Synchronized electrical stimulations, Ideal Binary masks (IdBMs). \\ \\

\noindent
\textbf{Abstract} Normal-hearing (NH) listeners have the ability to combine the audio input perceived by each ear to extract target information in challenging listening scenarios. Bilateral cochlear implant (BiCI) users, however, do not benefit as much as NH listeners do from a bilateral input. In this study, we investigate the effect that bilaterally synchronized electrical stimulation, bilaterally linked band selection, and ideal binary masks (IdBMs) have on the ability of 10 BiCIs to understand speech in background noise. The performance was assessed through a sentence-based speech intelligibility test, in a scenario where the speech signal was presented from the front and the interfering noise from one side. The linked band selection relies on the most favorable signal-to-noise-ratio (SNR) ear, which will select the bands to be stimulated for both CIs. \\ \\

\noindent
\textbf{Sensors in use:}
\begin{itemize}
    \item External cochlear microphone \\ \\
\end{itemize}

\noindent
\textbf{Summary} For individuals who have hearing loss in both ears, Bilateral Cochlear Implants (BiCI) are an option. A significant amount of research has focused on the correlation between band selection and speech intelligibility in unilateral CIs. Generally, the algorithm used for unilateral CIs is to choose the band with the lowest SNR. The same approach is used for BiCIs where each CI is treated as an individual device. As a result of this, the stimulation for each ear can be different result in unintelligible speech. This paper explores the advantages of linking band selection by conceding control of selection to the ear that has better SNR sound signals. Additionally, to improve linking the two CIs use synchronized clocks. Finally, Ideal Binary Mask algorithms are used to remove low-frequency noise in the same spectrum as the speech to improve sound quality for the patient. As a result, the mean percentage of recognized words for the linked and synchronized device was 15\% greater than the same value for the unlinked and unsynchronized BiCI.\\ \\

%%%%

\textbf{Paper III} 
\\ \\
\noindent
Quass, Gunnar Lennart, Peter Baumhoff, Dan Gnansia, Pierre Stahl, and Andrej Kral. “Level Coding by Phase Duration and Asymmetric Pulse Shape Reduce Channel Interactions in Cochlear Implants.” Hearing Research 396 (October 2020): 108070. https://doi.org/10.1016/j.heares.2020.108070. \\ \\

\noindent
\textbf{Keywords} Pulse phase loudness coding, Pseudomonophasic stimuli. \\ \\

\noindent
\textbf{Abstract} Conventional loudness coding with CIs by pulse current amplitude has a disadvantage: Increasing the stimulation current increases the spread of excitation in the auditory nerve, resulting in stronger channel interactions at high stimulation levels. These limit the number of effective information channels that a CI user can perceive. Stimulus intensity information (loudness) can alternatively be transmitted via pulse phase duration. We hypothesized that loudness coding by phase duration avoids the increase in the spread of the electric field and thus leads to less channel interactions at high stimulation levels. To avoid polarity effects, we combined this coding with pseudomonophasic stimuli. To test whether this affects the spread of excitation, 16 acutely deafened guinea pigs were implanted with CIs and neural activity from the inferior colliculus was recorded while stimulating with either biphasic, amplitude-coded pulses, or pseudomonophasic, duration- or amplitude-coded pulses. Pseudomonophasic stimuli combined with phase duration loudness coding reduced the lowest response thresholds and the spread of excitation. We investigated the channel interactions at suprathreshold levels by computing the phase-locking to a pulse train in the presence of an interacting pulse train on a different electrode on the CI. Pseudomonophasic pulses coupled with phase duration loudness coding reduced the interference by 4-5\% compared to biphasic pulses, depending on the place of stimulation. This effect of pseudomonophasic stimuli was achieved with amplitude coding only in the basal cochlea, indicating a distance- or volume dependent effect. Our results show that pseudomonophasic, phase-duration-coded stimuli slightly reduce channel interactions, suggesting a potential benefit for speech understanding in humans. \\ \\

\noindent
\textbf{Summary} Cochlear Implants (CIs) generally increase and decrease the amplitude of stimulation current to communicated respective changes in the loudness of sound collected by the microphone. A result of this is that for high volume signals, the current amplitude and subsequently the electomagnetic field around the channel increases which causes more channel interference. To avoid this, this paper recommends the use of phase pulse variation to code loudness. The result is that there is decreases current amplitude and hence less channel interference. Additionally, the channel interference that does exist is consistent across signal volume which makes it easier to eliminate via signal processing techniques. Testing of this alternative coding technique resulted in 4-5\% less channel interference.  \\ \\